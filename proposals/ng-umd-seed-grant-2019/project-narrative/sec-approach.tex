The goal of this work is to develop a framework for co-design that describes
aerial robotic design problems, defined as tuples of ''functionality space'',
''implementation space'', and ''resources space.'' To make the problem solvable, we will impose constraints on the resources: the SWAP constraints (such as the maximum size, weight and configuration of rotors) and constraints on the computing resources (maximum available memory). Given a set of tasks (functionality), we then can ask: {\it a) Is an implementation (involving software design and realization in embedded hardware) possible? b)  What is the implementation using minimal amount of information to solve the tasks given the resource constraints?}


%Our approach consists of two layers. Integrated modeling, analysis and
%optimization across these layers is a major  distinguishing aspect
%of our proposed research. 
%The first layer is concerned with the design of the drone hardware and sensors,
%including
%the frame of the drone, vision sensors, inertial measurement
%unit (IMU), and flight controller. The first layer is
%also concerned with the
%development of algorithms to achieve a set of mission tasks.
%The second layer is concerned with the realization of these algorithms in
%embedded processors and development of tools for the design and implementation
%of these multicore machine learning and signal processing applications.


Technical work involves: a) the  design  of  the  drone  hardware  and  sensors and the development of algorithms using the information from the sensors  to achieve a set of mission tasks; b) the realization of these algorithms in embedded processors and development of tools for the design and implementation of these multicore machine learning and signal processing applications. Integrated modeling, analysis and optimization across the different components  is  a  major  distinguishing  aspect  of  our  proposed  research

Figure~\ref{fig:flow} illustrates the new design methodology
that will be developed in the proposed project for the
specific Embodied AI application area of Drone-Level Codesign.
Major research efforts and deliverables
of the project are represented by the
blocks labeled Platform-aware algorithms, 
Parameterized libraries of algorithmic modules, 
Tools to map algorithms onto embedded processors,
and
Run-time system to coordinate algorithm configurations and hardware subsystems.
Our research will center around the integrated development of novel algorithms, 
libraries and software tools across these project components.


As a unifying formalism to promote systematic analysis and optimization across
the developed algorithms, libraries and tools, we will apply the paradigm of
dataflow-based modeling of signal and information processing systems (e.g.,
see~\cite{bhat2019x1}).
%\cite{lee1995x1, bhat2019x1}).
 Dataflow is widely-used in commercial,
domain-specific tools for signal and information processing, such as LabVIEW by
National Instruments, SystemVue by Keysight, and TensorFlow by Google. Our
prior research has influenced significant aspects of important commercial
dataflow tools (e.g., see~\cite{hsu2011x1, kee2012x2}). However, revolutionary
and strongly interdisciplinary advances are needed to dataflow methods to
address the complex challenges described in Section~\ref{sec:context}. We will
develop such advances as a core component of this project. More specifically,
our project will lead to new fundamental understandings among the areas of
dataflow, machine learning, and cyber-physical system design.  The results of
the proposed seed project will help to demonstrate our proposed vision of
Embodied AI with concrete simulation experiments in the context of Drone-level
Codesign.

\begin{figure}[h!]
\centering
\includegraphics[width=6.4in]
{figures/drone-codesign-flowchart.eps}
\caption {An illustration of the proposed new design methodology
for Embodied AI.}
\label{fig:flow}
\end{figure}

