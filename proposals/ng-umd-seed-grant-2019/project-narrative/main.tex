\documentclass[12pt]{article}
% \usepackage{cite}
\usepackage{epsfig}
\usepackage{url}
\usepackage{verbatim}
\usepackage[margin=1in]{geometry}
\usepackage{fancyhdr}
\usepackage{setspace}
% See https://texblog.org/2011/09/30/quick-note-on-line-spacing/


% \oddsidemargin=1.0in
% \evensidemargin=1.0in
% \textwidth=6.5in
% \topmargin=1.0in
% \textheight=9.0in
% \addtolength{\oddsidemargin}{-0.875in}
% \addtolength{\evensidemargin}{-0.875in}
% \addtolength{\textwidth}{1.5in}
% \addtolength{\topmargin}{-0.875in}
% \addtolength{\textheight}{1.5in}

% \setlength{\headheight}{15.2pt}
\pagestyle{fancy}
% \pagestyle{myheadings}

\lhead{NG-UMD Seed Grant Proposal}
\rhead{Co-design of Structure and Intelligence}
% \rfoot{\thepage}

\parindent=0.3in
\parskip=0in
% \pagestyle{plain} 
% \markright{hello}


\begin{document}

% Some useful instructions:
% \clearpage
% {\bf Quizzes:} 
% \gotable{0}{1}{rlp{.1in}rl}
% {\bf Instructor:} & Laura Taalman 		 
% && {\bf Office:}   & 020 Math/Physics \\
% {\bf Telephone:}  & 660-2829 (W), 220-1359 (H) 
% && {\bf E-mail:}   & taal@math.duke.edu \\
% \stoptable
% \input{macros}

\begin{center}
\begin{spacing}{1.8}
{\Large\textbf{Co-design of Structure and Intelligence
for Embedded System Optimization}}\par
\end{spacing}
\vspace{.06in}
{\sc  Proposal to the Northrop Grumman -- UMD Seed Grant Program}\\
\vspace{.06in}
Principal Investigators: Shuvra S.\ Bhattacharyya\textsuperscript{1,2} and 
Cornelia Ferm\"{u}ller\textsuperscript{1}\\
\vspace{0.06in}
1. Institute for Advanced Computer Studies (UMIACS)\\
2. Department of Electrical and Computer Engineering\\
\vspace{0.06in}
University of Maryland  of Maryland, College Park\\ 
Web: 
\url{http://users.umiacs.umd.edu/~fer/}, \url{http://www.ece.umd.edu/~ssb}\\
Email: \url{ssb@umd.edu}, \url{fer@umiacs.umd.edu}\\
\vspace{0.06in}
\today
\end{center}

\vspace{.14in}
\begin{abstract}

A challenge in modern engineering is to design complex embedded systems with
task specifications in a rigorous way with guarantees on correct operation, and
optimization of relevant implementation metrics, including processing speed,
power consumption, and memory requirements. We propose to develop foundations
for automated synthesis of embedded agents, and aerial robotics in particular.
Given a set of tasks and constraints on the size, weight and power of the
quadrotor, we propose to develop models and methods for synthesizing the
quadrotor platform. This includes the integrated selection of sensing and
embodied processing devices; the development of algorithms for control and
signal processing, which we study in the spirit of Active Perception as a
coupled entity; and the autogeneration of optimized software for embedded
devices. The developed algorithms and synthesized software
will be designed to reliably carry out a specified mission under
given constraints, and to adapt autonomously based on
changing characteristics in the operational environment. Through the novel models, methods and simulations developed in this
seed project, we will develop preliminary results and interdisciplinary
experience to support proposed new research on algorithms and software tools
for automated synthesis and optimization of embedded agents. 
%We will build on
%this foundation to develop proposals to major funding agencies, such as AFOSR,
%DARPA, and NSF.

\end{abstract}

\section{Context}
\label{sec:context}

The design of autonomous aerial vehicles such a quadrotors
requires many inter-related competencies, 
including 
the choice of  the aerial vehicle's structure and sensors; 
development and tuning of algorithms for sensor signal processing and knowledge extraction;
and design and implementation of embedded software that 
coordinates and executes these algorithms.
Conventional design methods integrate these competencies using
ad-hoc methods without systematic analysis or optimization
of the coupling across them. This leads to significant underutilization
of technologies, including machine learning techniques, 
sensors, actuators, and embedded processors, resulting
in devices that are significantly larger, less power efficient,
and provide less capability and less reliability compared
to what the underlying technologies and theory have the potential to provide.
Additionally, the ad-hoc design and integration processes greatly increase
the turnaround time to incorporate new technologies, which
slows down advancement of the state-of-the-art in aerial vehicle
systems.

% The following text can be incorporated when we have more space
\begin{comment}

Given size and weight, area and power constraints for a quadrotor
configuration, we ask: What is the minimal amount of information required to
solve a specific task under these constraints?  We conceptualize an autonomous
UAV (drone) as a collection of processes that allow it to perform a number of
behaviors or tasks. We will study algorithmic solution to these behaviors, by
considering the complexity of the representations involved, starting with
simple and progressing to more complex ones.

The very first competence is kinetic  stabilization (or egomotion estimation)
which is about maintaining a stable pose for the UAV, and it involves
estimating the quadrotor’s 6DOF pose (position and orientation) by combining
information from all the sensors on-board. The caveat here is that this has to
be done very fast and be reasonably robust to changes in environmental
conditions, and the speed at which this competency has to be performed
increases with a decrease in quadrotor size.  Next is the ability for obstacle
avoidance. Even for small drones at high-speed, the change in momentum is large
despite small weight – and this necessitates the need for high-speed robust
obstacle avoidance.  The next competence is homing, i.e., the capability to
find a specific location in an environment. Avoiding a reconstruction of the
scene, this involves maintaining a qualitative representation of a map.  In its
most basic form, one could just maintain a vector which points towards the home
(homing vector). We will study different representations and their
implementation in neural networks.  Even more complex are two other
capabilities. One is the ability to land (on a static or a dynamic surface).
This is required to dock onto a platform either for charging or safe landing. A
special case of landing would be to minimize crash impact or avoid certain
areas in case of an inevitable crash. This also involves failure tolerant
control. The other capability is to pursue or escape from other agents, which
involves prediction of their movements and online reactive control.

These above competences form a hierarchy, with each competence needing the one before it. It involves a sensorimotor loop which combines perception, planning, and control into one entity. Thus, it dictates a synergistic hardware and software architectural co-design for autonomous operation. 

Some of the hardware design questions which will be answered in this work are: What kind of cameras should we use – traditional frame-based or event-based? Should we use a cheap TOF sensor to accompany a camera? Should we use a micro-controller to pre-process the stereo stream to make the data bandwidth low? What is the best way to distribute computing in software and hardware?

\end{comment}


\section{Intellectual Merit}
\label{sec:merit}
To address the problem motivated in the previous section, we propose a holistic
investigation into the design and implementation of autonomous aerial vehicles.
In particular, we propose to develop algorithms, methodologies and software
tools for the design of quadrotors with different size and task constraints
that jointly consider the physical characteristics 
of the sensing devices, and embedded processing devices, along with embedded
software characteristics of the algorithms for machine learning, control and
signal processing, as they are realized on the embedded processing hardware.
This multidimensional optimization problem encompasses different strata,
including hardware, integrated chips, sensors, effectors and software -- the
set of programs running on the system, the   representations computed by these programs, and their efficient and reliable
implementation on the processing hardware. 

Holistic system modeling and optimization across these strata is a new research
area that has the potential to lead to disruptive design methodologies and
tools for aerial vehicles that serve mission-critical applications.  We call
this field ``Embodied AI''. While we propose to study Embodied AI in the
context of aerial vehicles, the concept has much broader applicability to other
types of cyber-physical systems.

We are interested in creating quadrotors with perception that autonomously
perform tasks in different environments.  The current approach in Robotics is
to use Computer Vision modules that aim to build a 3D representation of the
scene that is of general utility, generally with the so-called
simultaneous localization and mapping (SLAM) approach. Using this
representation, tasks are planned and accomplished to allow the quadrotor to
demonstrate autonomous behavior. However, SLAM approaches are computationally
very expensive, and  inefficient for aerial robots. To solve
many tasks, we don't need accurate 3D models.

We can take inspiration from biology. Insects and birds have solved the problem
of navigation and complex control without the need for building accurate 3D
maps of the scene. Their solutions are task driven.  Biological systems have
developed through evolution, and in this process their hardware (body, and
sensing devices) and software (brains) have co-evolved. For example, bees have
compound eyes, with a large field of view but of low resolution, which are
suited for high speed navigation with low computational cost --- as is evident
from the small amount of neurons in the bee's brain. Birds of prey, such as the
eagle, have large field of view eyes but also accurate vision to achieve both
agile flight and accurate recognition and tracking of prey, and this requires
much larger brains.  Similarly, in the engineering domain, while sensors like
LIDAR, RGB-D cameras or stereo cameras can make depth perception easier, they
may not be feasible on smaller robots due to Size, Weight, Area and Power (SWAP) 
constraints. Instead simpler representations when coupled directly with the control of the drone may be sufficient to solve certain tasks. For example, we may not need depth, but only image motion from lightweight cameras to avoid obstacles. 

The most distinguishing aspects of this research stem from our
interdisciplinary collaboration across the areas of computer vision, robotics,
and embedded signal processing.  In our work on platform-aware algorithms, and
parameterized libraries of algorithmic modules, we will build on our experience in computational motion analysis and visual servoing (e.g., see~\cite{ji2006x1,
barr2014x1, mitr2018x1}).
In our work on tools to map algorithms onto embedded processors, and run-time
techniques to coordinate algorithm configurations and hardware subsystems, we
will apply our experience in model-based architectures and design tools for
signal and information processing systems (e.g., see~\cite{bens2016x2,
bhat2019x1, bout2018x2}).





\section{Technical Approach}
\label{sec:approach}
The goal of this work is to develop a framework for co-design that describes
aerial robotic design problems, defined as tuples of ''functionality space'',
''implementation space'', and ''resources space.'' To make the problem solvable, we will impose constraints on the resources: the SWAP constraints (such as the maximum size, weight and configuration of rotors) and constraints on the computing resources (maximum available memory). Given a set of tasks (functionality), we then can ask: {\it a) Is an implementation (involving software design and realization in embedded hardware) possible? b)  What is the implementation using minimal amount of information to solve the tasks given the resource constraints?}


%Our approach consists of two layers. Integrated modeling, analysis and
%optimization across these layers is a major  distinguishing aspect
%of our proposed research. 
%The first layer is concerned with the design of the drone hardware and sensors,
%including
%the frame of the drone, vision sensors, inertial measurement
%unit (IMU), and flight controller. The first layer is
%also concerned with the
%development of algorithms to achieve a set of mission tasks.
%The second layer is concerned with the realization of these algorithms in
%embedded processors and development of tools for the design and implementation
%of these multicore machine learning and signal processing applications.


Technical work involves: a) the  design  of  the  drone  hardware  and  sensors and the development of algorithms using the information from the sensors  to achieve a set of mission tasks; b) the realization of these algorithms in embedded processors and development of tools for the design and implementation of these multicore machine learning and signal processing applications. Integrated modeling, analysis and optimization across the different components  is  a  major  distinguishing  aspect  of  our  proposed  research

Figure~\ref{fig:flow} illustrates the new design methodology
that will be developed in the proposed project for the
specific Embodied AI application area of Drone-Level Codesign.
Major research efforts and deliverables
of the project are represented by the
blocks labeled Platform-aware algorithms, 
Parameterized libraries of algorithmic modules, 
Tools to map algorithms onto embedded processors,
and
Run-time system to coordinate algorithm configurations and hardware subsystems.
Our research will center around the integrated development of novel algorithms, 
libraries and software tools across these project components.


As a unifying formalism to promote systematic analysis and optimization across
the developed algorithms, libraries and tools, we will apply the paradigm of
dataflow-based modeling of signal and information processing systems (e.g.,
see~\cite{bhat2019x1}).
%\cite{lee1995x1, bhat2019x1}).
 Dataflow is widely-used in commercial,
domain-specific tools for signal and information processing, such as LabVIEW by
National Instruments, SystemVue by Keysight, and TensorFlow by Google. Our
prior research has influenced significant aspects of important commercial
dataflow tools (e.g., see~\cite{hsu2011x1, kee2012x2}). However, revolutionary
and strongly interdisciplinary advances are needed to dataflow methods to
address the complex challenges described in Section~\ref{sec:context}. We will
develop such advances as a core component of this project. More specifically,
our project will lead to new fundamental understandings among the areas of
dataflow, machine learning, and cyber-physical system design.  The results of
the proposed seed project will help to demonstrate our proposed vision of
Embodied AI with concrete simulation experiments in the context of Drone-level
Codesign.

\begin{figure}[h!]
\centering
\includegraphics[width=6.4in]
{figures/drone-codesign-flowchart.eps}
\caption {An illustration of the proposed new design methodology
for Embodied AI.}
\label{fig:flow}
\end{figure}



\section{Alignment with Goals of Seed Grant Program}
\label{sec:alignment}

This seed project is aligned with the topic entitled {\em sensing, advanced
analytics, and/or responses to optimize embedded systems}.  Through our new
models and methods for Embodied AI, we will systematically integrate and
optimize resource, and accuracy-aware machine learning methods to enable much
smaller, more power efficient, and more reliable implementation of autonomous
aerial vehicles compared to what is possible using conventional design methods.
We will demonstrate these approaches through the design of smaller quadrotors
and new nano-quadrotors capable of monitoring the environment and performing
delivery tasks in a local environment. Their advantages over larger quadrotors
are due to safety, agility and power efficiency. Smaller quadrotors are safer
to humans, other beings and the environment, inflicting no or minimal damage
upon collision. They are more agile because  of smaller moment
of inertia, which makes their reaction to environmental factors faster.

The proposed seed grant would allow  jump
starting collaboration between the two PIs, Bhattacharyya and Ferm\"{u}ller,
and linking this collaboration to technical problems  of relevance to
Northrop Grumman.  The PIs both have significant experience in
interdisciplinary, multi-investigator collaborations, but have not worked together
before on joint research projects. The seed grant would enable development of
preliminary results that would allow the PIs to more effectively 
pursue funding opportunities at relevant agencies. Furthermore, research in collaboration with Northrop Grumman will help to strengthen the
proposals and research in terms of impact to industry and defense.






\section{Potential for Leveraging Proposed Work}
\label{sec:potential}
The proposed work integrates aspects of machine learning, cyber-physical
systems, computer vision, and electronic design automation. The unique
interdisciplinary nature of this work is enabled by the highly complementary
expertise of the investigators. The  theme of the work
provides significant potential for projects in major funding
agencies, including AFOSR, DARPA, and NSF. The relevance to aerial robotics,
embedded artificial intelligence, and cyber-physical systems, are major areas of interest at these agencies.
Specific examples of relevant proposal targets include DARPA's OFFSET BAAs and  Tactile Technology Office BAAs, and NSF's National Robotics Initiative and 
Cyber-Physical Systems Programs.
We will build on the preliminary results and
interdisciplinary experience enabled by this seed grant to pursue funding
opportunities at funding agencies such as these. 

PI Bhattacharyya and Ferm\"{u}ller are both US Citizens, which provides
potential to collaborate with Northrop Grumman on a broader
variety of topics, including those with citizenship restrictions.



% \section{Demonstration Plan}
% \label{sec:demo}
% \input{sec-demo}

% Figure example 1:
% \begin{figure}[h!]
% \centering
% \includegraphics[width=150mm]{figures/design-flow/design-flow.eps}
% \caption {Insert caption text here.}
% \label{fig:flow}
% \end{figure}

% Figure example 2:
% \begin{figure}
% \centering 
% \epsfig {file=./figures/network-diagram.pdf}
% \caption{Illustration of an application scenario.} 
% \label{fig:network} 
% \end{figure}	

\bibliographystyle{IEEEtran} 
\bibliography{refs}

% Some of these references are available online at \\
% \centerline{{\tt http://www.ece.umd.edu/DSPCAD/papers/contents.html}.}

\end{document}
