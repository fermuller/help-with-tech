
The design of autonomous aerial vehicles such a quadrotors
requires many inter-related competencies, 
including 
the choice of  the aerial vehicle's structure and sensors; 
development and tuning of algorithms for sensor signal processing and knowledge extraction;
and design and implementation of embedded software that 
coordinates and executes these algorithms.
Conventional design methods integrate these competencies using
ad-hoc methods without systematic analysis or optimization
of the coupling across them. This leads to significant underutilization
of technologies, including machine learning techniques, 
sensors, actuators, and embedded processors, resulting
in devices that are significantly larger, less power efficient,
and provide less capability and less reliability compared
to what the underlying technologies and theory have the potential to provide.
Additionally, the ad-hoc design and integration processes greatly increase
the turnaround time to incorporate new technologies, which
slows down advancement of the state-of-the-art in aerial vehicle
systems.

% The following text can be incorporated when we have more space
\begin{comment}

Given size and weight, area and power constraints for a quadrotor
configuration, we ask: What is the minimal amount of information required to
solve a specific task under these constraints?  We conceptualize an autonomous
UAV (drone) as a collection of processes that allow it to perform a number of
behaviors or tasks. We will study algorithmic solution to these behaviors, by
considering the complexity of the representations involved, starting with
simple and progressing to more complex ones.

The very first competence is kinetic  stabilization (or egomotion estimation)
which is about maintaining a stable pose for the UAV, and it involves
estimating the quadrotor’s 6DOF pose (position and orientation) by combining
information from all the sensors on-board. The caveat here is that this has to
be done very fast and be reasonably robust to changes in environmental
conditions, and the speed at which this competency has to be performed
increases with a decrease in quadrotor size.  Next is the ability for obstacle
avoidance. Even for small drones at high-speed, the change in momentum is large
despite small weight – and this necessitates the need for high-speed robust
obstacle avoidance.  The next competence is homing, i.e., the capability to
find a specific location in an environment. Avoiding a reconstruction of the
scene, this involves maintaining a qualitative representation of a map.  In its
most basic form, one could just maintain a vector which points towards the home
(homing vector). We will study different representations and their
implementation in neural networks.  Even more complex are two other
capabilities. One is the ability to land (on a static or a dynamic surface).
This is required to dock onto a platform either for charging or safe landing. A
special case of landing would be to minimize crash impact or avoid certain
areas in case of an inevitable crash. This also involves failure tolerant
control. The other capability is to pursue or escape from other agents, which
involves prediction of their movements and online reactive control.

These above competences form a hierarchy, with each competence needing the one before it. It involves a sensorimotor loop which combines perception, planning, and control into one entity. Thus, it dictates a synergistic hardware and software architectural co-design for autonomous operation. 

Some of the hardware design questions which will be answered in this work are: What kind of cameras should we use – traditional frame-based or event-based? Should we use a cheap TOF sensor to accompany a camera? Should we use a micro-controller to pre-process the stereo stream to make the data bandwidth low? What is the best way to distribute computing in software and hardware?

\end{comment}
